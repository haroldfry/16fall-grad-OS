%%%%%%%%%%%%%%%%%%%%%%%%%%%%%%%%%%%%%%%%%
% University/School Laboratory Report
% LaTeX Template
% Version 3.1 (25/3/14)
%
% This template has been downloaded from:
% http://www.Laemphlates.com
%
% Original author:
% Linux and Unix Users Group at Virginia Tech Wiki 
% (https://vtluug.org/wiki/Example_LaTeX_chem_lab_report)
%
% License:
% CC BY-NC-SA 3.0 (http://creativecommons.org/licenses/by-nc-sa/3.0/)
%
%%%%%%%%%%%%%%%%%%%%%%%%%%%%%%%%%%%%%%%%%

%----------------------------------------------------------------------------------------
%	PACKAGES AND DOCUMENT CONFIGURATIONS
%----------------------------------------------------------------------------------------

\documentclass{article}

\usepackage[margin=1.7in]{geometry}
\usepackage{siunitx} % Provides the \SI{}{} and \si{} command for typesetting SI units
\usepackage{graphicx} % Required for the inclusion of images
%\usepackage{natbib} % Required to change bibliography style to APA
\usepackage{amsmath} % Required for some math elements 
\usepackage{float}
\usepackage{hyperref}
\usepackage{algorithm, algpseudocode}% http://ctan.org/pkg/algorithms
\usepackage{algpseudocode}
\setlength\parindent{0pt} % Removes all indentation from paragraphs

%\usepackage{times} % Uncomment to use the Times New Roman font

%----------------------------------------------------------------------------------------
%	DOCUMENT INFORMATION
%----------------------------------------------------------------------------------------

\title{{\textbf{Virtual Machine vs. Container Runtimes: Performance Comparison}} \\
       \vspace{3\baselineskip}
       {\large Project proposal} \\
       \vspace{3\baselineskip}
       {\large Graduate Operating System} \\ 
       {\large CSE 60641} % Title
      }
%\author{John \textsc{Smith}} % Author name

\date{\today} % Date for the report
\author{Boyang Li, Bingyu Shen, Chao Zheng}

\begin{document}

\maketitle % Insert the title, author and date

\begin{center}
\begin{tabular}{l r}
Due:& September 22, 2016\\ 
\end{tabular}
\end{center}
\nocite{*}

\pagebreak

\section{Abstract}

Traditional virtualization machines that use hypervisors to virtualize hardware devices are 
enable heterogenous operating systems(OS) share limited computing resources. Meanwhile, sine each 
operating system need to be started from scratch, it imposes nonnegligible overheads. Recently, 
with the rise of OS level virtualization, it becomes possible to run various OS on a host system 
with very low overheads. In this paper, we compare various aspects of system performance between 
origin namespaces isolation, light weight container runtimes and virtual machines. We also run several
HPC workflows on Notre Dame disc cluster with Docker container runtime\cite{dockerwb}, Linux namespace
isolation environment, Makeflow\cite{albrecht2012makeflow} and Condor\cite{thain2003condor} to check the 
overheads of light weight container on cluster level.  

\section{Introduction}

\subsection{Linux Namespaces}

Providing services to a whole bunch of applications running on the system, it's important to provide different environment to each process accordingly and isolate those processes to prevent them from interfering with each other. Linux namespaces as one of the most basic component of Linux system provides such isolation and is the foundation of following Linux containers such as Docker, Singularity and more.

	Linux namespaces is a general term refers to many Linux namespaces, such as Process namespace, Network namespace, Mount namespace, User namespace and so on. In process namespace, all processes constitute a tree structure. A process can spin off a child process and the child process then becomes the root of a process tree. Processes in the parent namespace have the full view of its child namespace, while child namespace have no knowledge of the parent namespace. That's how the processes can be isolated from each other. It's similar in the network namespace, each process has access to entirely different set of networking interfaces. 

	Generally speaking, the basic idea of namespace is providing each process a restricted access to the system resources, so each process has a different view of the environment. By doing that processes are isolated from each other and therefore provides system security and stability under most circumstances. 

Linux Namespace\cite{rosen2013resource}

\subsection{Singularity}

Singularity\cite{singularity} Although namespace can offer certain extent of isolation, applications can only function within the host operating system and that significantly limits the usage of the computation resources. Containers are therefore generated. Containers realize virtualization in a fashion that guest operating systems share the Linix kernel of host operating system and retain their own image. Applications on top of the guest oeprating system therefore are quite close to the host operating system which guarantees the performance. 

Singularity is a container based on namespace technology. It's specifically designed for the purpose of mobility of compute. Singularity creates a container, examines the binary dependencies, and packages up library or system dependencies needed for guest operating system and applications, applications, and some other tolls into an executable package. Once done that, the container can move from system to system and only requires the Linux kernel of host operating system.

Singularity is motivated by HPC and therefore it's lightweight, easy to deploy and very portable. However,  

\subsection{Docker}

Docker\cite{dockerwb}

\subsection{Virtual Machine}

Virtual Machine\cite{rosenblum2005virtual}

\subsection{Virtualization on HPC}

\section{Related Work}

\section{Proposed Work}

\section{Evaluation}

We evaluate the performance of the lightweight container runtimes from micro and macro aspects. We conducting 
micro-benchmarks on a ThinkPad T430 machine, which consists of an Intel Core i5-3320M CPU with 16GB RAM, 256GB
SSD, 1GB Ethernet, running Ubuntu Linux 16.04 with Linux kernel 4.4.0-38-generic. For macro-benchmarks, we use
Notre Dame CRC disc cluster, The cluster consisted of twenty-four 8-core Intel Xeon E5620 CPUs each with 32GB
RAM, 12 2TB disks, 1Gb Ethernet, running Red Hat Enterprise Linux 6.8 with Linux kernel 2.6.32-642.4.2.el6.x86\_64.

\subsection{Microbenchmarks}

We are going to evaluate the various aspects of system performance between our VLLC, 
lightweight Linux container and virtual machine. We will apply \textbf{sysbench} to measure the CPU 
performance and memory bandwidth, \textbf{netperf} to evaluate the network throughput and \textbf{bonnie++} 
to evaluate disk I/O performance. We expect that the performance difference of CPU, memory and network would 
be negligiable. While, for the I/O performance we expect to see huge difference between Docker and Host 
system because of the use of \textbf{aufs} and less difference between Singularity, namespace isolation tool and
Host system because the first two configurations use local directory as drive.

\subsection{Macrobenchmarks}

Based on the previous work\cite{zheng2015integrating}, we plan to run macro-benchmarks on latest version of Makeflow, 
the HPC workflow management tools with HTCondor framework anduse Burrows-Wheeler Alignment (BWA) 
bioinformatics workflow, which consist 822 subtasks that can be highly parallelized. We will launch 
the workflow with VLLC, Docker and origin Makeflow system and compare the overall running time and per-task execution time.
The expected results would be Workflow launching with Docker can cause considerable overheads compare to workflow
running with VLLC and host system.

\medskip

Based on the benchmark results, we may be able to reach a conclusion that, lightweight container technologies 
provided certain level of resource virtualization but it also suffer certain system overheads depends on the level
of isolation they provided.

\section{Timeline}

%----------------------------------------------------------------------------------------
%	BIBLIOGRAPHY
%----------------------------------------------------------------------------------------
% print out all references without citing

\bibliographystyle{abbrv}

\bibliography{sample}

%----------------------------------------------------------------------------------------

\end{document}
